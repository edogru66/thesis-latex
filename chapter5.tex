\chapter{Conclusion and Discussion}
In this thesis, we present a method to discover paralog specific gene copy number within segmental duplications. Duplication-rich regions are mostly abandoned from 
structural variation discovery because of its complex nature. mrCaNaVaR was the first tool to address these problems, however it uses a sequence alignment file where reads are mapped to multiple locations. 

In our method, we utilized the singly unique nucleotides in order to differentiate the paralogs from each other. Our method was based on read depth and suffered from GC bias. We solved the problem using multiplicative version of LOESS method. We computed the average copy numbers of genes using read depth of singly unique nucleotides on that gene. And finally we sum the average copy numbers of genes residing in the same segmental duplication to find paralog specific copy number of genes.

Since read depth is limited to detecting duplications and deletions \cite{alkan2011genome}, we had to exclude other types of structural variations from our study. Furthermore, the absence of validated results makes us to compare our results with another method. Therefore, it was difficult for us to determine our method's performance. 

\section{Future Directions}

In this thesis, we applied our method to human genomes. It can be extended to genomes from different organisms. Furthermore, it can also be extended to different structural variation types. The latter requires a rather big change since we need to change our main signature (read depth) for structural variation discovery.

The method can be improved in many ways, however, there is an urgent need of a validated data set.