\chapter{Introduction}
- Early History (Darwin Mendel Identification of DNA etc.)

It all started with Darwin's book of The Origins of Species, which he published in 1859. Since then, the field of genomics have been paid a great deal of attention. What Darwin argues in his book is that animals that is more resilient to the environmental conditions contributed more to the genetic hereditary. His claim was based on the finches living in Galapagos Islands; those who survived are the ones that have developed some sort of adaptations according to the island they lived in. Therefore, Darwin concluded that the finches that have survived were naturally selected, what he called the process as natural selection.

In those years, there was an Augustinian monk whose name was Gregor Mendel was conducting experiments on his monastery's garden. His experiments were different than that of Darwin's, since Mendel used more scientific methods in his experiments. Mendel used pea plants to observe how the genetic hereditary passes down to younger generations. He crossbred the different pea plants each of which has different traits. Then he made two important observation. One of them is that some traits appear more than others and the second is even if a trait is no longer observed in one generation, it may reappear later in the offspring. His barely appreciated work at that time has earned him the title ``Father of Genetics".



- 1953 Watson and Crick model

- 1961 Marshall Nirenberg and Har Gobind Khorana DNA codons

- 1970 Frederick Sanger

- Human genome project

    - Preparation (1987 - 1990)

    - 1990 - 2001
    
    - Going on

\newpage
\section{Sequencing technologies}

\subsection{Early Sequencing Methods}
\subsection{High-Throughput Sequencing}
\section{Genomic Variation}
\subsection{Structural Variation}
\subsubsection{Copy Number Variation}
\section{Segmental Duplication}
\subsection{Singly Unique Nucleotides}
\section{Our Contribution}
\newpage
