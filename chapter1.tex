\chapter{Introduction}

It all started with Darwin's book of The Origins of Species, which he published in 1859. Since then, the field of genomics have been paid a great deal of attention. What Darwin argues in his book is that animals that is more resilient to the environmental conditions contributed more to the genetic hereditary. In those years, there was an Augustinian monk whose name was Gregor Mendel was conducting experiments on his monastery's garden. His experiments were different than that of Darwin's, since Mendel has done his experiments in a more systematic way. Mendel used pea plants to observe how the genetic hereditary passes down to younger generations. He crossbred the different pea plants each of which has different traits. Then he made two important observation. One of them is that some traits appear more than others and the second is even if a trait is no longer observed in one generation, it may reappear later in the offspring. His barely appreciated work at that time has earned him the title ``Father of Genetics".

A few years after the Mendel's experiments, Friedrich Miescher, a Swiss biochemist, have isolated what he called at that time ``nuclein" since it comes from the cell nucleus. In 1874, Miescher has successfully separated the ``nuclein" into protein and acid, and therefore the term ``nuclein" was later changed into nucleic acid, which is what we call it today\cite{pray2008discovery}.

In 1953, there was a breakthrough in the field of genomics with the discovery of double helix structure of DNA by James Watson and Francis Crick. Watson and Crick had answered the biggest question till then in genomics that could lead to solving the mystery about genetic heredity. In a series of articles published in several journals, Watson et al., apart from exposing the structure of DNA, have put their findings together with those of other researchers and come up with a mechanism for DNA replication along with three pieces of evidence supporting the complementary model which says that adenine has thymine and cytosine has guanine or vice versa, on the opposite strand of DNA \cite{watson1953structure}.

The complementary model opens the door of DNA sequencing since it allows the DNA to be reconstructed from one strand. There were some initial thoughts and methods on how to sequence the DNA, however, none of them were practical in terms of speed to sequence the whole genome. Frederick Sanger and Alan Coulson were the first who have developed the ``plus and minus" method, which is widely known as Sanger sequencing, for rapidly determining the order of smallest genetic units in DNA\cite{sanger1975rapid}. In an article published in 1970, Sanger et al. have set the gold standard in sequencing. They used DNA polymerase, an enzyme that is essential for DNA replication, to generate copy DNA sequences from the template DNA strand, all of which should start in the same exact position and end in normally one base larger (or smaller) than the previous. Sanger was able to stop the polymerase by removing an oxygen from the nucleotide at which he wants to stop. Sanger made these DNA sequences move at various speed under an electrical field. The speed of the DNA sequence is inversely proportional to its weight. Hence, from a fixed point of view, a light sensor will be able to detect the sequence of nucleotides since nucleotides will emit a light in different wavelengths.

As the technology in genomics advances, researchers were eyeing determining the sequence of all the bases in DNA. After two years of preparation and planning, Human Genome Project, an international research project, was started in 1990 with two simple goals. Along with the purpose of determining all the bases in human DNA, they had also wanted to find out the mapping of the genes in human genome. After eleven years from the start, 90\% of human DNA sequence was published. Since all the sequence comes from one individual, there can be a bias and certain genes could not be detected. Therefore, researchers were in need of what is called a ``reference genome". For this purpose, Genome Reference Consortium have built a human reference genome to provide the researchers with a good approximation of human DNA. It was composed of 7-8 individual DNA and updated regularly. 



- 1953 Watson and Crick model

- 1961 Marshall Nirenberg and Har Gobind Khorana DNA codons

- 1970 Frederick Sanger

- Human genome project

    - Preparation (1987 - 1990)

    - 1990 - 2001
    
    - Going on

\section{Sequencing technologies}

\subsection{Early Sequencing Methods}
\subsection{High-Throughput Sequencing}
\section{Genomic Variation}
\subsection{Structural Variation}
\subsubsection{Copy Number Variation}
\section{Segmental Duplication}
\subsection{Singly Unique Nucleotides}
\section{Our Contribution}
\newpage
